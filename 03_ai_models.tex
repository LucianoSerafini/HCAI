\section{Artificial Intelligent Models}
An AI agent takes decisions on how to act in the environment or how to
react to some input given by the humans and/or by the environment, on
the basis of one or more \emph{models} of the surrounding environment
and of the humans it is currently interacting with. Models are
abstract (computational) structures that allow to answer queries about
what holds in the current or in the past situation and to predict what
will be true/false in the future. Models can also be used to simulate
possible evolutions of the humans-environment in order to take the
``right'' decision now.

Artificial intelligent agents are equipped with a set of artificial
intelligence models that represent the knowledge of the agent about
the environment. Such a knowledge is used to support the agent in
making decisions about which action to perform. In general we cannot
assume that such models are \emph{correct}, i.e., that their
predictions are effectively true.  For this reason it is more
appropriate to speak about \emph{belief} instead of knowledge Neither
we can assume that models are \emph{complete} i.e., that they describe
the environment in all it's details. Indeed they are
\emph{simplified/abstract} representations of some aspect of the
environment obtained by abstracting away irrelevant (believe to be
irrelevant) details.

\subsection{Types of Models}

Providing a complete and coherent classification or even an ontology
of AI models is an ambitious goal, which probably is not reachable
within the time frame of this projects, however, we believe that an
initial classification is emerging from the AI research community. In
this classification we take a ``technological'' perspective. i.e., we
list different classes on the basis of the set of methodologies which
are used to specify, to represent, and to perform decision within this
model. In particular we will distinguish the following macro classes
of models.


\begin{itemize}
\item \emph{logical models:} the key aspects of the environment and
  the human are represented with a logical theory (set of formulas of
  a logical/formal language) and decision on the basis of this model
  are taken via logical reasoning. Examples of this type of model are
  Logical Knowledge Bases and Ontologies, Logic programs, classical
  planning domains specified via strips or some other action language,
  \dots.  Logical models are specified declaratively with a set of
  terms and formulas from a logic based requirement language. A good
  summary of all different logical models is provided in the book
  \cite{lifschitz2008handbook}. Logical models provide information of
  what is true, what is false and what logically follows from some
  premises.  To have an overview of all the type of models, how they
  can build and how they can used for decision making, planning and
  inference one can refer to the book \cite{minker2012logic}.

\item \emph{probabilistic models:} the key aspects of the environment
  and the human are represented by some probabilistic
  distribution. Decision are taken on the basis of probabilistic
  inference. Examples of this type of models are statistical graphical
  models (e.g., Bayesian networks, hidden Markov models, \dots).
  \begin{quote}
  \it The knowledge and beliefs of
cognitive agents are modeled using probability distributions
defined over structured systems of representation,
such as graphs, generative grammars, or
predicate logic. This development is crucial for making
probabilistic models relevant to cognitive science, where
structured representations are frequently viewed as
theoretically central. Second, the learning and reasoning
processes of cognitive agents are modeled using
advanced mathematical techniques from statistical
estimation, statistical physics, stochastic differential
equations, and information theory
\hfill \cite{chater2006probabilistic}
\end{quote}
Probabilistic models can be specified by directly providing the joint
probability density/mass function of the variables, or via
directed/undirected graphical models. Usually in probabilistic models
there are a set of observable variables (corresponding to the
data/evidences that an agent is capable to observe directly) and a set
of hidden variables, which distribution should be discovered from the
data.

The key concept in this type of model is the \emph{variable assignment},
i.e., an assignment to all the random variables, on which it is
possible to apply the model in order to predict the likelihood of such
an assignment.

\item \emph{Real Functional Models:} the key aspects of the worlds and
  the user are represented through (a set of) real functions that
  takes in input observable quantities and produces an estimation of
  some unobservable quantity. Examples of this type of models are
  Linear models, support vector machines, decision trees, random
  forest and (deep) neural networks.

  A large classo fo ``real funciton models'' is constituted by neural
  network models. A neural model is a graph of nodes. Each node is
  associated with a non linear activation function, the imput of a
  node n is a linear combination of the output of the function
  associated to nodes that precede n. Both the linear combination and
  the activation funciton are associated with a set of parameters,
  that need to be instantiated in order to fully define the
  function. A neural network model is also associated to a Loss (or
  Cost) function, that determines a cost in terms of the input and the
  output values of the network.

  The key concept of a neural network consists in the instantiation of the parameters.

  For every instantiation of it's parameters, a neural network
  computes a function $f:R^k\rightarrow R^h$ where $k$ is the number
  of "imput" nodes (i.e., the nodes that don't have any predecessor)
  and h is the number of the output nodes (i.e., the nodes which are
  not predecessor of any other node). The main objective in neural
  network is to find the instantiation of the parameters, that
  minimizes the Loss/Cost function.

\item \emph{Decision models} models the expected advantage an agent
  can obtain in following a certain policy (= action selection
  strategy). The tipical reward based model are Markov decision
  models, which constitute the basis of reinforcement learning. The
  key aspect of decision models is the fact that they represent the
  effect of taking actions in terms of reward for an agent. Both in
  reinforcement learning and game theory the main task in this model
  is to produce a policy that maximizes the expected rewards.

\item \emph{Optimization models} An optimization model has three main
  components: (i) An objective function. This is the function that
  needs to be optimized. (ii) A collection of decision variables. The
  solution to the optimization problem is the set of values of the
  decision variables for which the objective function reaches its
  optimal value. (iii) A collection of constraints that restrict the
  values of the decision variables.


\end{itemize}

In many cases a model presents charachteristics that are common to
more than one of the classes described above. We call them \emph{hybrid
  models}. Hybrid models are models that attempts to integrate some of
the previous type of models. Examples of such models are approaches
that integrates logical and numerical models (e.g., Logic Tensor
Networks, Lyrics, \dots) approaches that integrate logical and
statistical models (e.g., Markov Logic Networks and Probabilistic
Logic Programming), and finally approaches that integrate numeric,
statistical and logical models (e.g., DeepProbLog, deep probabilistic
logic programming, and Probabilistic Soft Logic)

