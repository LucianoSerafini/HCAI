\section{Related work}

\as{Should we have this section at the beginning of the paper rather than the end?}

\subsection{Human centered AI frameworks} 
\todo[inline]{add related work to \cite{wei2019toward}}
\todo[inline]{Add related work with Humane AI project}

\subsection{Ontologies for Human Centered Artificial Intelligence} 

\todo{maybe we can add some related work with the effort done in AI4EU
  and other projects to create an ontology for Human Centered
  Artificial Intelligence} 

AI4EU has developed an ontology that is used to describe the AI
resources availeble on the AI4EU platform \cite{ai4eu-D3.4}. 
The main concept introduced by the AI4EU ontology is the one of \emph{AI
  resource}. According to the deliverable an AI resource is:

\begin{quote}\it
\dots any entity that can be used for obtaining knowledge or
technology around AI. This can be specialized by various subclasses
that conceptualise different types of technology e.g., Dataset,
Software, Hardware, Ontology, etc. This can be specialized by various
subclasses that conceptualise different types of technology e.g.,
Dataset, Software, Hardware, Ontology, etc.
\end{quote}

There are also other attemps towards an ontology for Artificial
Intelligent Agent. One example is \cite{hawley2019challenges}. 
\todo{LS: Check \cite{hawley2019challenges} and see if there is something interesting} 

We can think that the main topic of dicussion of this paper is an
attempts to organize the subclasses of AI resources according to a
certain set of creatiria. In this sense what we have done in this
paper is orthogonal w.r.t. to the ontology defined in
\cite{ai4eu-D3}. A possible intersection between the vision
proposed in this paper, and the AI4EU ontology concerns also the
\emph{technical category}
tribute of AI-resources \texttt{AI\_Technical\_Category}. The possible
values of such an attributes are labels that refers to concepts
defined in this paper. For instance the labels ``computational logic'',
``deep learning'', ``constraints and sat'', ``probabilistic models'',
can be associated to an AI-resources that implements a logical model,
a functional model, a constrained modle and a probabilistic model,
respectively.

Apart from such a categorization it seems that the main concepts
introduced in this paper provides a description of AI-resources which
is complementary to the information provided by the AI4EU ontology.
Developing an ontology for AI-resources according to the decription of
this paper would be an interesting task and it could integrated with
the AI4EU ontology in order to complement the description of
AI-resources 

%%% Local Variables:
%%% mode: latex
%%% TeX-master: "main"
%%% End:
