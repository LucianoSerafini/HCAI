\section{Explainable, Verifiable and Collaborative AI systems} 

\subsection{Explainable AI} Explanable AI is the problem of providing
explanations to humans about some inference done by AI agent with it's
model. Depending on the model different explanations are possible:
For a survey of XAI see \cite{arrieta2020explainable}

A specific form of explanation is \emph{contrastive explanation}
\cite{miller2018contrastive} provides a formal framework for
contrastive explanation by extending \emph{structural causal models}
\cite{halpern2005causes1}.  The work define the notion of
\emph{contrastive explanation}, 
that is the explanation of an event relative to some contrast case, as
for instances explanations for questions like ``Why P rather than
Q?''. The proposed formalisation is intended to be general and
applicable to a wide rante of AI models.

\begin{description}
\item[Explanations of an inference in logical model] Though
  explanation are manly advocated for inference performed by deep
  neural networks, explanations are also necessary for inference
  performed in logical systems, in order to reach human centered
  artificial intelligence.

  Consider for instance the situation were an
  agent uses a SAT solver for answering the question $\phi$ posed on a
  knowledge base $K$. Suppose that $K\models\phi$. This implies that
  the agent will provide a positive (or negative) answer. But what
  about the explanation of such an answer?  The Solver will show that
  $K\cup\{\neg \phi\}$ is not satisfiable and for doing this it will
  search all the truth assignments without finding one that satisfies
  $K$ and $\neg\phi$. Showing the trace of the search would not
  constitute a good explanation. Alternatively the agent could provide
  a minimal set $K''subset K$ from which is it spossible to derive
  $\phi$ (i.e., to show that $K'\cup\{\neg \phi\}$ is
  inconsistent). Still the size of $K'$ might be very big and not
  understandable by a human. Furthermore, the encoding of the model in
  the set of axioms $K$ should not be very easy to read by a
  human. This means that an explanations should also contain the
  intuitive meaning of the langauge used to formula the knowledge base
  $K$. When the answer is negative i.e., if $K\not\models\phi$, then
  an explanation could be given in terms of a counter-model, i.e., an
  assignment that makes $K$ true and $\phi$ false. This is certainly
  mre readable than the previous case, but still the meaning of the
  axioms in $K$ should be conveyed to the human.

  A second method used for doing inference in logical model is to
  search for a deduction from the set of observations $O$ to the query
  $\phi$ (or $\neg\phi$), or viceversa find a proof from $\phi$ to $O$
  or from $\neg\phi$ to $O$. This is the case of explanation.
  In both cases an explanation is a sequence or a directed graph of
  formulas with $O$ and $\phi$ (or $\neg\phi$) at the extreme.

  In all the cases explanation is provided in terms of logical
  symbols. Therefore, to be understandable by humans, humans should be
  able to ascribe a meaning to these symbols. E.g, an explanation that
  contains the symbo $p124$ is probably not satisfactory. 
  
\item[Explanations of inference in probabilistic model] One of the
  basic inference task in probabilistic model is called \emph{Most
    Probable Explanation (MPE)}. MPE is the problem of finding a
  complete assignment of values to a set of variables that has the maximum
  likelihood given some variables that are instantiated with some
  observations. Formally the MPE is the following problem:
  $$
  \argmax_{x}Pr(X=x\mid E=e)
  $$
  where $E$ and $X$ are disjoint set of random variables. Intuitively
  the solution of the MPE is a value of $X$ that better explains the
  observation $e$. This is a rather crude explanation, that does not
  provide any rational argument behind the final decision of the
  model. A more sophisticated notion of explanation of the inference
  in a statistical model is offered by considering causal models
  \cite{pearl2009causality}, interventions, and counterfactuals.
  \cite{chajewska2013defining,halpern2005causes1,halpern2005causes2,%
    sprenger2018foundations,peters2017elements}.  In this case
  explanation are defined as \dots (TO BE FINISHED)

\item[Explanations of an inference in real function model]
  The massive interest on explainability has surged due to the success
  of deep neural network. Various form of explanations has been
  proposed for the conclusion obtained from a DNN.  A very general
  definition of explanation of the decision of a real function can be
  defined with the help of causal models.
  In particular, in \cite{chattopadhyay2019neural} a deep neural
  network is seen as a Structural Causal Model \cite{pearl2009causality}

  An alternative approach, which consider only the imput/output pair
  of the model is by determining the contribution of each single imput
  feature to the final prediction by using additive feature
  attributions. 

  \begin{definition}[Additive feature attributions]
      Suppose $f:\mathcal{X} \rightarrow\mathbb{R}$
is a model mapping an $M$-dimensional feature space $\mathcal{X}$ to realvalued
predictions. Additive feature attributions for $f(\mathbf{x})$ at
input $\mathbf{x}=(x_1,\dots,x_M)\in\mathcal{X}$, 
comprise of a reference (or baseline) attribution $\phi_0$ and a set
of feature attributions $\phi_1,\dots,\phi_n$ such that $f(\mathbf{x}) =
\phi_0+\sum_{i=1}^M\phi_i$. 
\end{definition}
According to the previous definition $\phi_i$ is the contribution of
the feature $x_i$ to the decition $f(\mathbf{x})$. 


  
\item[Explanations of an inference in action model] Action models
  include planning model, reinforcement learning models and game
  theoretic models. For each of the type of model a specific notion of
  explanations has been developed. In the following we quickly analyse
  them.
  
  \paragraph{Explanations in Planning} The approaches and the
  challenges of explanation in planning are described in
  \cite{chakraborti2020emerging}, which defines the different forms of 
  \emph{explanations of a plan or a policy as a solution of a given
    planning problem}. A planning problem is composed of a planning
  domain and a desired property (goal + optimality condition, \dots).
  The solution of a planning problem is a plan or more generally a
  policy i.e., a function $\pi: S\rightarrow A$ that select the
  actions to execute if the agent is in a specific state.
  \begin{align}
    \label{eq:plannin-decision}
  \Pi, \pi\models \tau
  \end{align}
  means that, $\tau$ can be achieved by following the policy $\pi$
  in the planning domain $\Pi$. 
  An explanation of the decision \eqref{eq:plannin-decision} 
  is an artifact $\mathcal{E}$ ``understandable by the explanee'' such that
  $$
  \mathcal{E},\pi\models_u\tau 
  \mbox{ and }
  \mathcal{E},\pi'\not\models_u\tau
  \mbox{ 
    for all $\pi'$ different from $\pi$.}
  $$
  The content of the
  explanation $\mathcal{E}$ can vary greatly depending on the needs of
  the explainee.
  
  \paragraph{Explainable reinforcement learning}
  For a survey in explanations in Reinforcement Learning see
  \cite{puiutta2020explainable}. We report here two approaches that
  provide a clear and formal definition of explanation in
  reinforcement learning. 
  
    \cite{sequeira2019interestingness} proposes an explainable reinforcement learning
    (XRL) framework that analyzes an agent’s history of interaction
    with the environment to extract interestingness elements that help
    explain its behavior. In a nutshell the explanation of the
    behavious of the agent is provided in terms of four aggregations
    values about the interaction the agent has with the enviroment.
    In particular suppose that the agent interaction is the following
    sequence
    $$
    t= z_0,a_1,r_1,z_1,a_2,r_2,\dots z_{n-1},a_n,r_n,z_n
    $$
    they propose to ``explain'' the behavious of the agent in terms of
    Frequency, Transition Value, Execution Certainty, Sequences, which
    can be computed starting from the trace $t$.

    From the same data (trace $t$) \cite{madumal2019explainable} learn
    a causal model as defined in \cite{halpern2005causes1} from which
    they are capable to provide counterfactual explanations of why an
    action has (not) been executed in a certain state $s$.  The causal
    model is a DAG where state variables are linked trough actions,
    intuitively $X\stackrel{\rightarrow}{A} Y$ where $X$, and $Y$ are
    state variables and $A$ an action, suggests that the value of $Y$
    after the execution of $A$ depends on the value of $X$.  The
    leaves nodes of the DAG correspond to the rewards.  The
    explanation of ``why the action $a$ in state $s$'' is given in
    terms of the values of the variables entering in the action $A$,
    and the values of the variables entering in the final reward.  The
    explanation of ``why not action $b$ in state $s$'' is given in
    terms of conterfactual explanation. I.e., by comparing the
    explanation of $a$ in $s$ and the explanation of $b$ in $s$.
    
    \paragraph{Explanation in Game theory}
    No specific method has been found for explaining decisions taken by
    game theoretic models. Instead game theoretic approaches have been
    used to provide explanations in other models. 
    \cite{merrick2019explanation} proposes a general game theoretic
    method to generate explanations based on additive feature
    attributions.

  \item[Explanations of an inference in optimization model] One of the
    most important application domain of optimization is
    scheduling. Explanation is scheduling has been defined in
    \cite{vcyras2019argumentation} via argumentation.  The aurthors
    formally define argumentative explanations as to why a given
    schedule is not ‘good’. We can define the framework very
    abstractly as follows; Let $M$ be a makespan problem and $S$ be an
    optimal and feasible schedule for $M$. Then we write
    $$
    M,S \models O \wedge F \wedge D
    $$
    where $O$ is a formula expressing the optimality condition w.r.t.,
    a partial order $<_S$ on the set of schedules. 
    $F$ is a formula expressing the feasibility condition imposed in $M$
    and $D$ if a formula expressing the desiderata (from the user),
    i.e., a subset of schedules. 
    The proposed approach provide explanations for 
    $$
    M,S\not\models O 
    $$
    by finding a schedule $S'\neq S$ with better perfornmance w.r.t.,
    $S$.
    The explanation of 
    $$
    M,S\not\models F
    $$
    By finding the minimal $M',S'\subseteq M,S$ such that
    $M',S'\not\models F$. Similarly, the explanation of
    $$
    M,S\not\models D
    $$
    is provided by finding a minimal subpart of $S'\subseteq S$
    such that for all $S'\not\subseteq S''$ for all $S''\in O$> 

    {\color{red}FROM MICHELE \tt

      1) Visto che la maggior parte dei sistemi di ottimizzazione si
      basano su modelli definiti da esperti, il problema della
      spiegazione è decisamente meno sentito.

      Un esperto non ha in generale difficoltà ad interpretare una
      soluzione. Anche nel caso il modello sia un po’ convoluto
      (capita in SAT e MILP in particolare), è generalmente facile
      presentare una sua soluzione in termini naturali per un esperto.

      2) Parecchio lavoro è stato dedicato allo spiegare l’ottimalità
      o l’infeasibility (i.e. perché una certa soluzione è ottima, o
      cosa la rende infeasible).

      Questi temi però non li trovi associati al termine
      “explainability”, perché è nato molto più tardi. Si parla invece
      di:

      - Irreducible Infeasible Sets (in MILP)
      
      - Conflicts (in CP e SAT)
      
      - reduced costs (come  prova di ottimalità in LP)
      
      - Minimal Critical Sets (in scheduling)
      

      Addirittura c’è gente che usa questi approcci per spiegare le
      predizioni delle deep network (dopo averle tradotte in modelli
      simbolici).

In sintesi:

1) Il temine “explainability” non è di sicuro molto usato in contesto di ottimizzazione
2) Il problema di  spiegare modelli e soluzioni non è molto sentito, se non in alcuni contesti specifici (vedi sopra).
3) Detto questo, qualcosa potenzialmente da fare ci sarebbe. Per esempio, i vari approcci per l’individuazione di conflitti etc. hanno tradizionalmente una finalità tecnica: i  conflitti estratti sono utili per i solver, ma non necessariamente chiari per un  essere umano.
    }
  \end{description}
  
\subsection{Verifying AI models}

\rb{Should we only aim for covering deductive software verification?} % possibilities below:

%deductive verification
%model checking
%general proof assistants
%correctness by construction
%SMT solvers
%runtime verification
%software testing
%coverage criteria for testing
%static analysis
%dynamic analysis
%...
%formidable challenges lie ahead with respect to verifiable AI...!



\todo{add relations between safety and liveness properties of
  different AI models \dots Raul}
  
% single-agent (the AI agent) and non-deterministic environments, sensors and actuators. This is what we have also in the “onion”, i.e., observations (produced by sensors) and actions (Produced by actuators); if humans are taken into account. They act independently from the artificial agent, and therefore they are a source of indeterminism in the environment. Explain how this leads us into safety and liveness properties, and notions of fairness; temporal logics may be applied as a means to specify such requirements. humans interact with the environment and with the agent, in a feedback loop.

Verifiable AI consists of the set of methods, techniques and tools
that support checking key properties of an AI system or a larger
system that includes AI as one or more key components.  In this part
we concentrate only in the verification of the AI component of a
system, i.e., the AI model. In general we can distinguish between
functional and non functional requirements. Non functional
requirements specifies how the sistem should be rather than what it
should do. Since we consider `abstract AI models'' we focus on
verification of functional requirements on how the model and behaves
in all it's phases. 
  
 \begin{description}
 \item[Verifying logical models] The behavior of a logical system is
   encoded in the logic itself, and once one is able to encode the
   conditions that the system should comply in terms of a set of pairs
   of formulas $\left<\phi_o,\phi_a\right>$ where $\phi_o$ are the
   properties of the input observation and $\phi_a$ encodes the
   property of the decision/action taken by the system, then the one
   can check if 
   $$
   M_L, \phi_o \models \phi_a
   $$
   and it will be guaranteed that whenever the model receives in input
   an observation that satisfies $\phi_o$, (i.e., $O\models \phi_o$)
   then the output generated will necessarily be an action that
   satisfies (or is consistent with) $\phi_a$.
   
 \item[Verifying probabilistic models]
   Formal verification of bayesian networks \cite{pmlr-v72-shih18a}
   Probabilistic model checking \cite{kwiatkowska2018probabilistic}

 \item[Verifying numerical models] According to
   \cite{kohli2019towards}, there are three main groups of
   verification tasks Adversarial Testing, Robust learning, and Formal
   Verification. There are many works in the area of formal
   verification of neural networks and other numerical model.
   However, one of the most important element of the models which are
   learned via supervision, are the properties of the training data.
   Important aspects to be verified are also the fact that the system
   don't use certain critical data to do the prediction.

   \cite{narodytska2018formal} discusses a set of interesting
   properties of neural network, including properties that relate
   inputs and outputs of the network, e.g. robustness and
   invertibility, and properties that relate two networks, like
   network equivalence. For instance one property of the neural
   network is global $\epsilon$-robustness. 

   A feedforward neural network $f:\R^n\rightarrow \{0,1\}^m$ is
   globally $\epsilon$-robust if for any $x,\tau\in\R^n$ with
   $||\tau||_{\infty} \leq \epsilon$, $f(x+\tau)=f(x)$. 
   
 \item[Verifying action models]
   model checking \cite{clarke2018introduction} and
   probabilistic model checking \cite{kwiatkowska2018probabilistic}
   Formal verification of reinforcement learning
   \cite{van2017challenges}
   
 \item[Verifying optimization models]
   
\end{description}

\subsection{Co.AI}
\begin{description}
\item[Collaboration with agents with logical models]
\item[Collaboration with agents with probabilistic models]
\item[Collaboration with agents with numerical models]
\item[Collaboration with agents with action models]
\item[Collaboration with agents with optimization models]    
\end{description}
