\section{Introduction}

% \paragraph{Agent-Environment pair} 


\todo{\footnotesize Luca: I would mention humans immediately in the reference framework, otherwise the human-centered part of the concept will not be clear. What about an agent-human-environment framework? in which human is at the center?

Alessandro: I agree.  If we propose "foundations for human-centered AI" we should first say what we mean by "human-centerd AI", and then explain why foundations are needed for it.

Jaques: In the definition we can cite the HLEG AI guidelines of Trustworthy AI

Luciano: I have tried to address your observations in this first page}

In the recent years, artificial intelliget (AI) systems become popular
and have an important and pervasive impact in the personal life and in
the society evolution.  The fact that AI systems can significatively
affect human lives, and can play an active role in the society,
fostered the discussion on how these new systems should behave in
relationship with humans.  Human Centered AI (HCAI) concerns the study
of how present and future AI systems will interact with human lives in
a mixed society composed of artificial and human equally intelligent
agents.
Due the interdisciplinarity of the topic, the discussion involves both
technical and non technical people which have see AI agent from
different perspectives which are in many cases difficult to integrate
in a coherend vision in which each point of view can contribute in and
complement other perspectives.
Of particular interests are the recent attempts to provide
requirements and ethical guidelines that regulates Human Centered
Artificial Intelligent agents. Such a guidelines describes HCAI agents
that are at a very high level, nonetheless such a guidelines has
certainly an important impact on the future of AI. However,
understanding the impact of such a high level qualitative requirements
on the technical development an HCAI agents is neither trivial nor
unique.

The main objective of this document is to provide a reference
description of the main components and functions of HCAI Agent that
helps to bridge the gap between the high level specifications of HCAI
agents provided by e.g., the European Commission in
\cite{eu-ethical-guidelines} which is close enough to the real
technical implementation of an HCAI agent.
Such a description should contribute to the grounding of high level
qualitative statements into technical requirements on a HCAI agent.
The necessity of such a grounding is explicitly state in

\begin{quote}
Requirements for Trustworthy AI should be ''translated'' into
procedures and/or constraints on procedures, which should be anchored
in the AI system’s architecture. \cite[page 21]{eu-ethical-guidelines}.
\end{quote}

In other words, the main objective of this paper is an attempt to fill
the gap between ``non technical description of HCAI agent'' and more
``technical structure'' that is understandable without a deep
knowledge of the main technics of AI, such as Machine Learning,
Automated Reasoning, Reinforcement Learning, Probabilistic inference
and optimisation.  Despite, this we will try to be precise enough to
allow the mapping of hight level concepts that one can find for
instance in HCAI ethical guidelines into a more technical and
technically operative description.  We are aware of the limits of this
paper, and we believe that this is just a first step towards the
filling of such a cultural gap.

\section{Definition of HCAI agent}
Instead of given a single definition of HCAI Agent, in wat follows we
list a set of features that AI agent should meet in order to be
considered Human Centered.

\paragraph{Agent-environment pair}
\def\agent{\mathbf{Ag}}
\def\agentState{State(\agent)}
\def\environment{\mathbf{Env}}
\def\environmentState{State(\environment)}

We consider a reference framework composed of an HCAI agent (or simply
agent) that operates in an environment in presence of humans. We call
this framework the agent-environment framework. HCAI concentrates on
environments in which humans directly interacts with the artificial
agent in a collaborative attitude.  In an agent-environment framework,
there is a clear division between what is internal to the agent, and
what is external to the agent, i.e., the environment.
Both the agent and the enviroment at every time poin are in a state,
that we refer as the state of the agent and the state of the
environment. Since we don't consider quantum physic both agents and
environments are in one single state at every time. While the internal
state of the agent might be directly accessible to the agent, the
state of the environment and the humans that inabitate it is sont
directly accessible by the agent, which can only have partial
observations on it.
The agent-environment pair can be represented as a pair $(\agent,\environment)$ with
$\agent=\{\agentState_t\}_{t\geq t_o}$ and $\environment = \{\environmentState_t\}_{t\geq t_0}$ where $t_0$ is
    some orign point. Notice that in the environment $\environment$
    includes also the humnas that interacts with the agent.

  \paragraph{Observations and actions}
The agent observes the environment via sensors, and acts upon the
environment via effectors. Every interaction between the agent and the
environment happens through these two categories. Observations and
actions are considered in a very broad sense. Observation or percept,
Sensors provides the agent with observations or precepts that range
from low level data, natural language, images, movies, etc.  Actions
may range from physical actions, as moving ahead, moving an arm,
grasping an object, to communicative actions (speech acts), such as
displaying some data on the screen, uttering a sentence, smiling, or
playing a song or a movie. In general there is no synchronisation
between actions and observations, and both actions and observations
can happen independently without following a precise protocol.  HCAI
imposes a number of restrictions and requirements on the data that can
be observed by an agent and the actions that can be
executed. Concerning observations, HCAI systems ``\dots must guarantee
privacy and data protection throughout a system’s entire lifecycle'',
and they should be unbiased. Concerning the actions that an HCAI is
allowed to execute the major requirement is that they are
``\dots consistent with the input, and that the decisions [to execute
action] are made in a way allowing validation of the underlying
process''. Furthermore HCAI should execute only legal actions
i.e., actions complying with all applicable laws and regulations. 

\paragraph{Goal directed agents} 
AI agent always have one or more goals to achieve \footnote{The
  so-called \emph{reflex} agents \cite{aima4}, which internal
  architecture reduces to a single reasoning component that directly
  maps observations to actions do not internally maintain an explicit
  representation of their goals. Even in this case, the mapping
  function or rule base that implements their reflexes are
  nevertheless programmed or machine learned with some implicit goal
  in mind for these agents to achieve.}. Agent goals can be codified
in many different ways, depending on the architecture of an agent. For
instance the goal of the agent could be to optimize a certain function
or to find the best path to achieve a position, or to improve it's
capability of recognising objects in the scene, or to correctly answer
users queries. In HCAI the achievement of such a goal should not harm
the humans present in the environment. Making explicit such goals in a
necessary condition to suppor the principles of prevention of harm and
fairness mentioned in \cite{eu-ethical-guidelines}.  \footnote{From
  \cite[page 12]{eu-ethical-guidelines} ``AI systems should neither
  cause nor exacerbate harm or otherwise adversely affect human
  beings, [\dots] ensuring that individuals and groups are free from
  unfair bias, discrimination and stigmatisation. If unfair biases can
  be avoided, AI systems could even increase societal fairness.''}

\paragraph{No absolutely autonomous agents} HCAI agent are not
completely autonomous, since their behaviour should always be determined by
some external environmental or human event. Humans should always be
allowed to intervene in order to avoid that the agent will perform
one specific or a set of actions. This is necessary to guarantee ``The
principle of respect for human autonomy'', that allows a human to
autonomously decide to prevent the execution of one or more actions to
the HCAI agent. From the architectural poin of view it means that
among the set of observations that an HCAI agent has to employ, there
should be a sensor that detect the willing of the humans to interrups
some planned actions.

\section{Schema for HCAI agent}
A pictorial representation of our vision of a HCAI agent is shown in
figure~\ref{fig:hcai-onion}. This picture interprets literally the adjective 
``human centered'' by drawing the human, and his/her environment, at
the center of the scene surrounded by the agent. This is the opposite
of the usual representation that
show an AI agent in the center surrounded by the environment and the humans.
A first observation concerns the fact
that humans cannot be separated from the environment where they
operate. Indeed they are \emph{part} of the environment; they are
\emph{embedded} in it, In other words, humans and environment are
complementary, interconnected, and interdependent in the natural
world, and they interrelate to one another. Therefore HCAI Agents should
take into account both humans and the environment where humans
operate. In the picture this is represented by the blue and green Yin
Yang. 
\begin{figure}[h]
  \begin{center}
      \begin{tikzpicture}\sf
  \draw[fill=yellow!40] (0,0) circle [radius=2.75];  
  \draw[fill=red!30] (0,0) circle [radius=2];
  \draw[white,line width=10pt] (-2,0) to (2,0);
  \draw[white,line width=10pt] (0,-2) to (0,2);
  \draw (0,0) circle [radius=2];
  \foreach \x in {1,...,4}
    \coordinate (a\x) at (\x*360/4:1.5);
  \foreach \x in {1,...,5}
    \coordinate (b\x) at (\x*360/5-130:2.25);
    \draw[fill=blue!70!yellow!35] (0,0) circle [radius=1.25];
    \draw[fill=green!40!black] (0,1.25) arc (90:270:1.25);
    \draw[draw=none,fill=green!40!black] (0,.6125) circle [radius=.6125];  
    \draw[draw=none,fill=blue!75!yellow!30] (0,-.6125) circle [radius=.6125];
    \node[circle,fill=green!40!black] at (0,.6125){\color{blue!75!yellow!30}\LARGE H};
    \node[green!40!black] at (0,-.5) {\LARGE E};
  \draw[draw=none,postaction={decorate,decoration={text along path,text align=center,text={Requirement}}}] (a2) to [bend left=45] (a1);
    \draw[draw=none,postaction={decorate,decoration={text along path,text align=center,text={Explanation}}}] (a3) to [bend left=45] (a2);
   \draw[draw=none,postaction={decorate,decoration={raise=-5,text along path,text align=center,text={Observation}}}] (a3) to [bend right=45] (a4);
   \draw[draw=none,postaction={decorate,decoration={text along path,text align=center,text={Action}}}] (a1) to [bend left=45] (a4);
 \draw[draw=none,postaction={decorate,decoration={text along path,text align=center,text={System}}}] (b2) to [bend left=40] (b1);
   \draw[draw=none,postaction={decorate,decoration={text along path,text align=center,text={Intelligent}}}] (b3) to [bend left=40] (b2);
   \draw[draw=none,postaction={decorate,decoration={text along path,text align=center,text={Artificial}}}] (b4) to [bend left=40] (b3);
   \draw[draw=none,postaction={decorate,decoration={text along path,text align=center,text={Integrated}}}] (b5) to [bend left=40] (b4);
 \end{tikzpicture}

\end{center}
\caption{\label{fig:hcai-onion} 
A simple
  schematization of Human-Centered Artificial Intelligence Agent 
 (external ring) and its interactions with
  humans (H) and environment (E) (inner ring).}
\end{figure}

The external ring in Figure~\ref{fig:hcai-onion} represents the
artificial intelligent agent(s). It can refer to one single
intelligent agent or a set of interrelated intelligent agents, which
autonomously interact following, e.g., a multi-agent paradigm. Every
HCAI agen accomplish some task directly relevant for the humans it
interacts with.

using one (or more) specific approaches/models. For instance an
``artificial reasoner'' could represent its knowledge in some logical
formalism and support query answering and inference through automatic
reasoning (e.g., SAT, ASP); an ``artificial classifier'' could be
implemented in a deep neural network that is capable to classify
images into different classes. An ``artificial planner'' can produce
plans and take decision exploiting classical planning techniques or
reinforcement learning. To solve complex tasks, different systems and
methodologies should be integrated. A black box integration of each
``intelligent agent'' is not sufficient; it is necessary to integrate
and make all these different approaches to collaborate one another in
a glass-box method. 

Finally, the intermediate ring represents the interaction between the
human/environment and the integrated artificial intelligent system.
The type of interactions that one can see between humans/environment and
artificial agents happen across a set of artifacts that are
``shared'' by the humans/environments and the artificial system.
We briefly describe them here, but we will describe more extensively
in the rest of the document:
\emph{Dependability Requirements} are
artifacts, produced by humans, that specify the expected behaviour and
some other non-functional properties of the artificial intelligent
system. Some of them are usually specified in a formal language (that
has an intuitive semantics for the human) so that it is possible to
verify automatically that the artificial intelligent system behaves
according to the requirement to a certain certainty degree.  Some
of them are not expressed in a formal language and how they can be
represented in a mathematical structure is an open issue.
\emph{Actions} are considered in a broader sense. They
  represent the actions that the artificial agent can perform as well
  as the action that the human can perform in the direction of the
  artificial agent. They can be physical actions (that has effects on
  the environment) or informative actions (that have effects on the
  knowledge of the human or of the artificial agent). 
\emph{Observations} are all the data that the artificial
  intelligent system can collect through its sensors.
\emph{Explanations:} are artifacts that are produced by an artificial
  agent that ``explains'' to the human the reason of its
  behaviour. E.g., the reason why it took an action or a
  decision. Explanations should be human understandable and acceptable
  in a rational system shared by the machine and the human. 

%%% Local Variables:
%%% mode: latex
%%% TeX-master: "main"
%%% End:
